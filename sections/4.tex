a) Since $X$ and $Y$ are independent:

\begin{align*}
  P(X \leq 2\ \texttt{and}\ Y \leq 2) &= P(X \leq 2) P(Y \leq 2) \\
                                      &= (P_X(1) + P_X(2))(P_Y(1) + P_Y(2)) \\
                                      &= (\dfrac{1}{4} + \dfrac{1}{8}) (\dfrac{1}{6} + \dfrac{1}{6}) = \dfrac{1}{4}
\end{align*}

b) From Demorgan's law:

\begin{align*}
  \overline{A \cup B} = \bar{A} \cap \bar{B}
\end{align*}

Hence

\begin{align*}
  \overline{(X > 2 \cup Y > 2)} = \overline{(X > 2)} \cap \overline{(Y > 2)} = (X \leq 2) \cap (Y \leq 2)
\end{align*}

and using the identity

\begin{align*}
  P(A) = 1 - P(\bar{A})
\end{align*}

we get that

\begin{align*}
  P(X > 2 \cup Y > 2) = 1 - P(X \leq 2, Y \leq 2) = 1 - \dfrac{1}{4} = \dfrac{3}{4}
\end{align*}

c) Since $X$ and $Y$ are independent, knowing $Y$ does not give us information about $X$:

\begin{align*}
  P(X > 2 / Y > 2) = P(X > 2) = P_X(3) + P_X(4) = \dfrac{1}{8} + \dfrac{1}{2} = \dfrac{5}{8}
\end{align*}

d)

\begin{align*}
  X < Y = &\{(X = 1, Y = 2), (X = 1, Y = 3), (X = 1, Y = 4), \\
          &(X = 2, Y = 3), (X = 2, Y = 4), (X = 3, Y = 4)\}
\end{align*}

And $X,Y$ are independent, hence

\begin{align*}
  P(X < Y) &= P_X(1)P_Y(2) + P_X(1)P_Y(3) + P_X(1)P_Y(4) \\
           &+ P_X(2)P_Y(3) + P_X(2)P_Y(4) + P_X(3)P_Y(4) \\
           &= \dfrac{7}{24}
\end{align*}
